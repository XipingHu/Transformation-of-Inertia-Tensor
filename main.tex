\documentclass[10pt,aspectratio=169,mathserif]{beamer}		
%设置为 Beamer 文档类型,设置字体为 10pt,长宽比为16:9,数学字体为 serif 风格

%%%%-----导入宏包-----%%%%
\usepackage{cgcl}			%导入 CCNU 模板宏包
% \usepackage{ctex}			 %导入 ctex 宏包,添加中文支持
\usepackage{amsmath,amsfonts,amssymb,bm}   %导入数学公式所需宏包
\usepackage{color}			 %字体颜色支持
\usepackage{graphicx,hyperref,url,mathtools}	
%%%%%%%%%%%%%%%%%%	


\beamertemplateballitem		%设置 Beamer 主题

%%%%------------------------%%%%%
\catcode`\。=\active         %或者=13
\newcommand{。}{.}				
%将正文中的“。”号转换为“.”。
%%%%%%%%%%%%%%%%%%%%%

%%%%----首页信息设置----%%%%
\title{Transformation of  Inertia Tensor}
\subtitle{}			
%%%%----标题设置


\author[Xiping Hu]{
  Xiping Hu \\\medskip
  {\small \url{XipingHu@hust.edu.cn}} \\
  {\small \url{https://github.com/XipingHu}}}
%%%%----个人信息设置
  
\institute[IOPP]{
	School of Physics\\
  Huazhong University of Scienec and Technology}
%%%%----机构信息

\date[Mar. 21 2018]{\today
%  2016年12月01日
}
%%%%----日期信息
  
\begin{document}

\begin{frame}
\titlepage
\end{frame}				%生成标题页

\section{Outline}
\begin{frame}
\frametitle{Outline}
\tableofcontents
\end{frame}				%生成提纲页


\section{Introduction to the Inertia Tensor}
\begin{frame}
  \frametitle{Introduction to the Inertia Tensor}
	  \begin{block}{The form of Inertia Tensor}
	    We write the Inertia Tensor as:
	    \begin{equation*}
	    \begin{aligned}
		    I&=
		    \begin{bmatrix}
		    \sum\limits_\alpha m_\alpha \left( x_{\alpha , 2}^2 + x_{\alpha , 3}^2 \right)  & -\sum\limits_\alpha m_\alpha x_{\alpha , 1}x_{\alpha , 2} & \sum\limits_\alpha m_\alpha x_{\alpha , 1}x_{\alpha , 3} \\
		    -\sum\limits_\alpha m_\alpha x_{\alpha , 2}x_{\alpha , 3} & \sum\limits_\alpha m_\alpha \left( x_{\alpha , 1}^2 + x_{\alpha , 3}^2 \right) & \sum\limits_\alpha m_\alpha x_{\alpha , 2}x_{\alpha , 3} \\
		    -\sum\limits_\alpha m_\alpha x_{\alpha , 3}x_{\alpha , 1} & -\sum\limits_\alpha m_\alpha x_{\alpha , 3}x_{\alpha , 2} & \sum\limits_\alpha m_\alpha \left( x_{\alpha , 1}^2 + x_{\alpha , 2}^2 \right) \\
		    \end{bmatrix}\\
		    &\coloneqq\begin{bmatrix}
		    I_{11} & I_{12} & I_{13} \\
		    I_{21} & I_{22} & I_{23} \\
		    I_{31} & I_{32} & I_{33}
		    \end{bmatrix}
	    \end{aligned}
	    \end{equation*}
	    Which is symmetric since
	    \begin{equation*}
	    I_{ij} = I_{ji}
	    \end{equation*}
	  \end{block}
\end{frame}
\begin{frame}
	\frametitle{Introduction to the Inertia Tensor}
	\begin{block}{The Rotational Kinetic Energy}
		\begin{equation*}
		T_{rot} = \sum\limits_{i,j} \frac{1}{2} \omega_i I_{ij} \omega_j = \frac{1}{2} \bm{\omega}' \bm{I} \bm{\omega}
	\end{equation*}
	To make life much easier, we may find an axis in which the cross terms vanish.
	\end{block}
\end{frame}

\section{Diagonalization of symmetric matrix}
\begin{frame}
	\frametitle{Diagonalization of symmetric matrix}
	\begin{block}{Some facts on Symmetric Matrices}
		\textbf{Theorem:}\hspace{1pt} Any symmetric matrix
		\begin{enumerate}
			\item has only real eigenvalues
			\item is always diagonalizable
			\item has orthogonal eigenvectors
		\end{enumerate}
	\end{block}
\end{frame}
\begin{frame}
	\frametitle{Diagonalization of symmetric matrix}
	\begin{block}{Find eigenvalue and eigenvectors}
		\textbf{Example (p113):}\hspace{3pt}
		\begin{equation*}
		\bm{I}=
		\begin{bmatrix}
		\vspace{5pt}
		\frac{2}{3} \beta & -\frac{1}{4} \beta & -\frac{1}{4} \beta \\
		\vspace{5pt}
		-\frac{1}{4} \beta & \frac{2}{3} \beta & -\frac{1}{4} \beta \\
		-\frac{1}{4} \beta & -\frac{1}{4} \beta & \frac{2}{3} \beta
		\end{bmatrix}
		\end{equation*}
		Our aim: Find $\bm{I^*}$, which is diagonal and similar with $\bm{I}$\\
		\vspace{3pt}
		Solve
		\begin{equation*}
		\left|  \bm{I}-\lambda \bm{E} = 0\right| 
		\end{equation*}
		Whereas $\bm{E}$ represents the Elementary Matrix
	\end{block}
\end{frame}
\begin{frame}
	\frametitle{Diagonalization of symmetric matrix}
	\begin{block}{}
		\begin{equation}
		\label{eq:1}
		\left| 
		\begin{matrix}
		\vspace{5pt}
		\frac{2}{3} \beta -\lambda & -\frac{1}{4} \beta & -\frac{1}{4} \beta \\
		\vspace{5pt}
		-\frac{1}{4} \beta & \frac{2}{3} \beta -\lambda & -\frac{1}{4} \beta \\
		-\frac{1}{4} \beta & -\frac{1}{4} \beta & \frac{2}{3} \beta - \lambda
		\end{matrix}
		\right| =0
		\end{equation}
		Equation \ref{eq:1} can be simplified as:
		\begin{equation*}
		\left( \frac{11}{12} \beta - \lambda \right) \left( \frac{11}{12} \beta - \lambda \right) \left( \frac{1}{6} \beta - \lambda \right) 
		\end{equation*}
		The eigenvalues of $\bm{I}$ are
		\begin{equation*}
			\lambda_1 = \lambda_2 = \frac{11}{12} \beta
		\end{equation*}
		\begin{equation*}
			 \lambda_3 = \frac{1}{6} \beta 
		\end{equation*}
	\end{block}
\end{frame}
\begin{frame}
	\frametitle{Diagonalization of symmetric matrix}
	\begin{block}{}
		To find eigenvectors, insert $\lambda_i$ into $\left(  \bm{I}-\lambda_i \bm{E}\right) \bm{\omega}=0$\\
		\vspace{10pt}
		for $\lambda_1=\lambda_2 =\frac{11}{12}\beta$
		\begin{equation*}
		\left[
		\begin{matrix}
		\vspace{5pt}
		\frac{2}{3} \beta -\frac{11}{12}\beta & -\frac{1}{4} \beta & -\frac{1}{4} \beta \\
		\vspace{5pt}
		-\frac{1}{4} \beta & \frac{2}{3} \beta -\frac{11}{12}\beta & -\frac{1}{4} \beta \\
		-\frac{1}{4} \beta & -\frac{1}{4} \beta & \frac{2}{3} \beta - \frac{11}{12}\beta
		\end{matrix}
		\right]\left[ 
		\begin{matrix}
		\omega_1\\
		\omega_2\\
		\omega_3
		\end{matrix} \right] =0
		\end{equation*}
		
		\begin{equation*}
		\begin{aligned}
		\omega_1+\omega_2+\omega_3&=0\\
		\omega_1+\omega_2+\omega_3&=0\\
		\omega_1+\omega_2+\omega_3&=0
		\end{aligned}
		\end{equation*}
		\begin{equation*}
		\bm{\omega}=\left[ 1,-1,0 \right]' \hspace{5pt} \text{or} \hspace{5pt} \bm{\omega}=\left[ 1,0,-1 \right]'
		\end{equation*}
	\end{block}
\end{frame}
\begin{frame}
	\frametitle{Diagonalization of symmetric matrix}
	\begin{block}{}
		for $\lambda_3=\frac{1}{6}\beta$
		\begin{equation*}
		\left[
		\begin{matrix}
		\vspace{5pt}
		\frac{2}{3} \beta -\frac{1}{6}\beta & -\frac{1}{4} \beta & -\frac{1}{4} \beta \\
		\vspace{5pt}
		-\frac{1}{4} \beta & \frac{2}{3} \beta -\frac{1}{6}\beta & -\frac{1}{4} \beta \\
		-\frac{1}{4} \beta & -\frac{1}{4} \beta & \frac{2}{3} \beta - \frac{1}{6}\beta
		\end{matrix}
		\right]\left[ 
		\begin{matrix}
		\omega_1\\
		\omega_2\\
		\omega_3
		\end{matrix} \right] =0
		\end{equation*}
		
		\begin{equation*}
		\begin{aligned}
		-2\omega_1+\omega_2+\omega_3&=0\\
		-2\omega_1+\omega_2+\omega_3&=0\\
		-2\omega_1+\omega_2+\omega_3&=0
		\end{aligned}
		\end{equation*}
		\begin{equation*}
		\bm{\omega}=\left[ 1,1,1 \right]' 
		\end{equation*}
	\end{block}
\end{frame}
\begin{frame}
	\frametitle{Diagonalization of symmetric matrix}
	\begin{block}{Gram-Schmidt process}
		\begin{equation*}
		\mathrm {proj} _{\mathbf {u} }\,(\mathbf {v} )={\langle \mathbf {v} ,\mathbf {u} \rangle  \over \langle \mathbf {u} ,\mathbf {u} \rangle }{\mathbf {u} }
		\end{equation*}
		\begin{equation*}
		\begin{aligned}\mathbf {u} _{1}&=\mathbf {v} _{1},&\mathbf {e} _{1}&={\mathbf {u} _{1} \over \|\mathbf {u} _{1}\|}\\\mathbf {u} _{2}&=\mathbf {v} _{2}-\mathrm {proj} _{\mathbf {u} _{1}}\,(\mathbf {v} _{2}),&\mathbf {e} _{2}&={\mathbf {u} _{2} \over \|\mathbf {u} _{2}\|}\\\mathbf {u} _{3}&=\mathbf {v} _{3}-\mathrm {proj} _{\mathbf {u} _{1}}\,(\mathbf {v} _{3})-\mathrm {proj} _{\mathbf {u} _{2}}\,(\mathbf {v} _{3}),&\mathbf {e} _{3}&={\mathbf {u} _{3} \over \|\mathbf {u} _{3}\|}\\\ \ \vdots &&{}\ \ \vdots \\\mathbf {u} _{k}&=\mathbf {v} _{k}-\sum _{j=1}^{k-1}\mathrm {proj} _{\mathbf {u} _{j}}\,(\mathbf {v} _{k}),&\mathbf {e} _{k}&={\mathbf {u} _{k} \over \|\mathbf {u} _{k}\|}.\end{aligned}
		\end{equation*}
	\end{block}
\end{frame}
\begin{frame}
	\frametitle{Diagonalization of symmetric matrix}
	\begin{block}{}
		for $\lambda_3=\frac{1}{6}\beta$
		\begin{equation*}
		\bm{\omega}=\frac{1}{\sqrt{3}}\left[ 1,1,1 \right]' 
		\end{equation*}
		for $\lambda_1=\lambda_2 =\frac{11}{12}\beta$
		\begin{equation*}
		\bm{\omega}=\frac{1}{\sqrt{3}}\left[ -\sqrt{\frac{3}{2}},\sqrt{\frac{3}{2}},0\right]' 
		\end{equation*}
		or
		\begin{equation*}
		\bm{\omega}=\frac{1}{\sqrt{3}}\left[ -\sqrt{\frac{3}{2}},-\sqrt{\frac{1}{2}},\sqrt{2} \right]' 
		\end{equation*}
	\end{block}
\end{frame}
\begin{frame}
	\frametitle{Diagonalization of symmetric matrix}
	\begin{block}{}
		\begin{equation*}
		\bm{\Lambda}=\frac{1}{\sqrt{3}}
		\begin{bmatrix}
		\vspace{5pt}
		1 &-\sqrt{\dfrac{3}{2}}&-\sqrt{\dfrac{1}{2}}\\
		\vspace{5pt}
		1 &\sqrt{\dfrac{3}{2}}&-\sqrt{\dfrac{1}{2}}\\
		1 &0&\sqrt{2}
		\end{bmatrix}
		\end{equation*}

		\begin{equation*}
		\bm{I^*}=\bm{\Lambda 'I \Lambda}=
		\begin{bmatrix}
		\vspace{5pt}
		\dfrac{1}{6}\beta &0&0\\
		\vspace{5pt}
		0&\dfrac{11}{12}\beta&0\\
		0 &0&\dfrac{11}{12}\beta
		\end{bmatrix}
		\end{equation*}
	\end{block}
\end{frame}
\begin{frame}
	\frametitle{Diagonalization of symmetric matrix}
	\begin{block}{}
		\begin{equation*}
		\bm{\omega' I \omega}=\bm{\omega' \Lambda \Lambda' I \Lambda \Lambda' \omega}
		=\bm{\left( \Lambda' \omega\right)'\left(  \Lambda' I \Lambda \right) \left( \Lambda' \omega\right) }
		=\bm{(\omega^*)' I^*\omega^* }
		\end{equation*}
		\begin{equation*}
		\bm{\omega^*}=\bm{\Lambda' \omega}
		\end{equation*}
	\end{block}
	\begin{block}{Why $\Lambda'$ not $\Lambda$ ?}
		$\bm{\omega}$ is the base vector of coordinates.
	\end{block}
\end{frame}

\section{References}
\begin{frame}{References}
	\begin{block}{References}
		\begin{thebibliography}{99} 
			\bibitem{} Gao Deng Dai Shu. Weisheng Qiu
			\bibitem{} Classical MEchanics (Third Edition), Herbert Goldstein
		\end{thebibliography}
	\end{block}
\begin{block}{This Document is placed under GFDL 1.3}
	\href{https://www.gnu.org/licenses/fdl-1.3.en.html}{GNU Free Document License Version 1.3 by Free Software Foundation}\\
	See also: https://www.gnu.org/licenses/fdl-1.3.en.html
\end{block}
\begin{block}{Source code of this document}
		https://github.com/XipingHu/Transformation-of-Inertia-Tensor.git
\end{block}
\end{frame}

\end{document}